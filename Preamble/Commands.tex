% Preamble/Commands.tex
% -------------------------------------------------------------------------------
% Commands and macros for the report

% Added commands ----------------------------------------------------------------
\newcommand{\VE}{\vphantom{E^E}}
\renewcommand{\hat}{\widehat}                           % Better hats
\renewcommand{\tilde}[1]{\widetilde{#1}}                % Better tilde
\renewcommand{\Tilde}[1]{\widetilde{#1}}                % Better tilde
\renewcommand{\epsilon}{\varepsilon}                    % Better epsilon
\renewcommand{\sim}{\thicksim}                          % Better sim

% Calculus
\newcommand{\e}{\mathrm{e}}                             % Eulers number
\newcommand{\ind}{\mathbbm{1}}                          % Indicator-function
\renewcommand{\d}{\mathrm{d}}                           % For integrals
\newcommand{\re}[1]{\mathrm{Re}\left(\VE#1\right)}      % Real part
\newcommand{\im}[1]{\mathrm{Im}\left(\VE#1\right)}      % Imaginary part

% Linear algebra
\renewcommand{\vec}[1]{\bm{#1}}                         % Bold vectors
\newcommand{\hvec}[1]{\hat{\vec{#1}}}                   % Bold hat vectors
\newcommand{\Det}[1]{\mathrm{det}\left(\VE#1\right)}    % Determinant
\newcommand{\Tr}[1]{\mathrm{Tr}\left(\VE#1\right)}      % Trace
\newcommand{\inner}[2]{\langle\VE#1,#2\rangle}          % Inner product
\newcommand{\Span}[1]{\mathrm{span}\left(\VE#1\right)}  % Span
\newcommand{\range}[1]{\mathrm{range}\left(\VE#1\right)}% Range
\newcommand{\Dim}[1]{\mathrm{dim}\left(\VE#1\right)}    % Dimension
\newcommand{\norm}[1]{\left\|\VE#1\right\|}             % Norm

% Probability
\renewcommand{\P}[1]{\mathbb{P}\left(\VE#1\right)}      % Probability function
\newcommand{\E}[2][]{\mathbb{E}_{#1}\left[\VE#2\right]} % Expected value
\newcommand{\Var}[1]{\mathrm{Var}\left[\VE#1\right]}    % Variance
\newcommand{\Cov}[1]{\mathrm{Cov}\left[\VE#1\right]}    % Covariance
\newcommand{\Cor}[1]{\mathrm{Cor}\left[\VE#1\right]}    % Correlation

% Time Series Analysis
\newcommand{\wn}[1]{\mathrm{wn}\left(\VE#1\right)}      % White noise
\newcommand{\AR}[1]{\mathrm{AR}\left(\VE#1\right)}      % Autoregressive
\newcommand{\ARMA}[1]{\mathrm{ARMA}\left(\VE#1\right)}  % Autoregressive moving average
\newcommand{\VAR}[1]{\mathrm{VAR}\left(\VE#1\right)}    % Vector Autoregressive
\newcommand{\MA}[1]{\mathrm{MA}\left(\VE#1\right)}      % Moving Average

% Finance
\newcommand{\CO}[1]{\mathrm{C}\left(\VE#1\right)}      % Call option
\newcommand{\PO}[1]{\mathrm{P}\left(\VE#1\right)}      % Put Option
\newcommand{\IV}[2][]{\mathrm{IV}_{#1}\left(\VE#2\right)}      % Implied Volatility


% Mathematical symbols ----------------------------------------------------------
\newcommand{\N}{\mathbb{N}}                             % Natural numbers
\newcommand{\Z}{\mathbb{Z}}                             % Integers
\newcommand{\Q}{\mathbb{Q}}                             % Rational numbers
\newcommand{\R}{\mathbb{R}}                             % Real numbers
\newcommand{\C}{\mathbb{C}}                             % Complex numbers
\newcommand{\F}{\mathbb{F}}                             % Complex and real numbers

\newcommand{\bigO}{\mathcal{O}}                         % Big O-notation
\DeclareMathAlphabet{\pazocal}{OMS}{zplm}{m}{n}         % Use other math format






% Theorems, definitions etc. ----------------------------------------------------
% http://www.ctex.org/documents/packages/math/amsthdoc.pdf
\newtheoremstyle{Thm}
  {0.5em}{0em}{\itshape}{}{\bfseries}{.}{\newline}
  {\thmname{#1}\,\thmnumber{#2\ifx#3\empty\else:\fi}\ifx#3\empty\else\,\,\fi\thmnote{#3}}
\theoremstyle{Thm}                                    % Bold title, italic body
\newtheorem{theo}{Theorem}[chapter]
\newenvironment{thm}[1]
    {\begin{thmenv}\begin{theo}#1}{\end{theo}\end{thmenv}}
 
\newtheorem{lemm}{Lemma}[chapter]
\newenvironment{lem}[1]
    {\begin{thmenv}\begin{lemm}#1}{\end{lemm}\end{thmenv}}
    
\newtheorem{propo}{Proposition}[chapter]            
\newenvironment{prop}[1]
    {\begin{thmenv}\begin{propo}#1}{\end{propo}\end{thmenv}}
    
\newtheorem{corol}{Corollary}[chapter]
\newenvironment{coro}[1]
    {\begin{thmenv}\begin{corol}#1}{\end{corol}\end{thmenv}}

\newtheoremstyle{Defn}
  {0.5em}{0em}{}{}{\bfseries}{.}{\newline}
  {\thmname{#1}\,\thmnumber{#2\ifx#3\empty\else:\fi}\ifx#3\empty\else\,\,\fi\thmnote{#3}}
\theoremstyle{Defn}                               % Bold title, normal text
\newtheorem{defi}{Definition}[chapter]                  
\newenvironment{defn}[1]
    {\begin{defnenv}\begin{defi}#1}{\end{defi}\end{defnenv}}
    
\newtheorem{exam}{Example}[chapter]
\newenvironment{exmp}[1]
    {\begin{exmpenv}\begin{exam}#1}{\end{exam}\end{exmpenv}}

\newtheorem*{remark}{Remark}

\renewcommand\qedsymbol{$\blacksquare$}





% Figure-macros -----------------------------------------------------------------
% Image figure
\newcommand{\imgfig}[3][0.75]{
  \begin{figure}[H]
    \centering
    \includegraphics[width=#1\textwidth]{#2}
    \caption{#3}
    \label{Figures:#2}
  \end{figure}
}

% Double image figure
% https://en.wikibooks.org/wiki/LaTeX/Floats,_Figures_and_Captions#Subfloats
\newcommand{\dimgfig}[5][0.5]{
  \ifx\dimgleftwidth\undefined
    \newlength{\dimgleftwidth}
    \newlength{\dimgrightwidth}
  \fi
  \setlength{\dimgleftwidth}{#1\textwidth-0.02\textwidth}
  \setlength{\dimgrightwidth}{0.96\textwidth-\dimgleftwidth}
  \begin{figure}[H]
    \centering
    \begin{minipage}[t]{\dimgleftwidth}
      \centering
      \includegraphics[width=\linewidth]{#2}
      \caption{#3}
      \label{Figures:#2}
    \end{minipage}
    \hfill
    \begin{minipage}[t]{\dimgrightwidth}
      \centering
      \includegraphics[width=\linewidth]{#4}
      \caption{#5}
      \label{Figures:#4}
    \end{minipage}
  \end{figure}
}