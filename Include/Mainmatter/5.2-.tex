\section{Risk-Neutral Density}\label{Sec.App:RND}
As the terms in \eqref{Eq:RND_Derivatives} have all been determined, it is now possible to determine the risk-neutral density by rewriting \eqref{Eq:PartialC} to
\begin{align}\label{Eq:App_RND}
    \exp\left(r\tau\right)\frac{\partial^2}{\partial K^2}C(t,K) &= f^\Q(S_T\mid S_t),
\end{align}
under the condition that $K=S_T$. Recall that we set the risk-free interest rate to $r=0.0368$. As we already know the current stock price of GOOG ($S_t=100$), the only variable in \eqref{Eq:App_RND} yet to be set is the time to maturity. For the following two examples we set $\tau=0.02$ and $\tau=0.07$. This yields the density curves illustrated in \autoref{Figures:Figures/Pictures/APP1/Densiteter_GOOG.pdf}.

\imgfig[1]{Figures/Pictures/APP1/Densiteter_GOOG.pdf}{Risk-neutral density for current stock price $S_t=100$ and time to maturity $\tau=0.02$ (left) and $\tau=0.07$ (right).}

In \autoref{Figures:Figures/Pictures/APP1/Densiteter_GOOG.pdf} we observe that for $\tau=0.07$ the risk-neutral density closely resembles a bell-shaped density with mean around $100$ which is ATM. This is to be expected as the discounted stock price is a martingale under the risk-neutral measure, that is
\begin{align*}
    \E[\Q]{\exp(-r\tau)S_T\mid\mathcal{F}_t} = \exp(-r\tau)S_t.
\end{align*}
As both $\tau$ and $r$ are relatively small $\exp(-r\tau)$ almost becomes negligible. For $\tau=0.02$ we observe that the risk-neutral density is not as smooth or closely resembles that of a bell-shaped density. This also results in its integration value of $0.715$ being lower that the one for $\tau=0.07$ which is $0.896$. These results were rather unexpected as we can see in \autoref{Figures:Figures/Pictures/KDist.pdf} that there is more data available for $\tau\approx0.02$ compared to $\tau\approx0.07$. Hence, it was expected that the neural network would perform better for $\tau = 0.02$, which should be reflected in the risk-neutral density.

Furthermore, it is evident from both plots depicted in \autoref{Figures:Figures/Pictures/APP1/Densiteter_GOOG.pdf} that as the stock price at maturity deviates from the current stock price, the density becomes more distorted or fluctuates more significantly. This is especially seen on the left side of the densities. This phenomenon can be attributed to insufficient data during the training of the neural network at these specific points or the numerically differentiation. When the neural network predicts values for inputs with limited or no prior knowledge, it is more prone to making inaccurate predictions, which is reflected in the risk-neutral density.

In general, when we integrate the risk-neutral densities for current stock price $S_t=100$ and a given time to maturity it integrates to less than 1. As is especially evident in \autoref{Figures:Figures/Pictures/APP1/Densiteter_GOOG.pdf} (left), the density's left side is missing mass. Thus, we expect options with strike prices well below ATM to be impacted greatly by this missing mass, which will presumably result in lower option prices. The reason for this is that when calculating the option prices by \eqref{Eq:RND_Pricing}, the lower bound of the integral is the strike price. Contrarily, we do not expect the options with strike prices above ATM to be as greatly impacted as the right side of the density looks decent in its approximation. However, we do not know whether the density is a good approximation of the real one, and hence how these will exactly affect the prices. 

The neural network which has been used until now is using ReLU as the activation function. ReLU is, as mentioned, not differentiable neither ones or twice. This can possibly explain the irregularities and sudden jumps in the densities, as the first and second partial derivatives of the neural network function is used for calculating the densities. Hence, we instead try using the sigmoid function as the activation function in the neural network which is approximating the implied volatility function. As mentioned in \autoref{Sec.App:NN}, when using the other activation functions than ReLU, the performance of the network worsens, which for sigmoid gives a MAPE on the test set of $4.53\%$. However, it still approximates the implied volatility fairly well, as seen in \autoref{Figures:Figures/Pictures/APP1/sigmoid_surface.png}, and hence we will try and calculate the densities for $\tau=0.02$ and $\tau=0.07$, as in \autoref{Figures:Figures/Pictures/APP1/Densiteter_GOOG.pdf}, using the neural network with the sigmoid function as the activation function. For these time to maturities and the neural network with the sigmoid function as the activation function, we calculate the risk-neutral densities seen in \autoref{Figures:Figures/Pictures/APP1/dens_sigmoid_GOOG.pdf}.

\imgfig[1]{Figures/Pictures/APP1/dens_sigmoid_GOOG.pdf}{Risk-neutral density for stock price $S_t=100$ and time to maturity $\tau=0.02$ (left) and $\tau=0.07$ (right) using the neural network with the sigmoid as the activation function.}

In \autoref{Figures:Figures/Pictures/APP1/dens_sigmoid_GOOG.pdf} it is clearly seen that the densities are much more smooth than those in \autoref{Figures:Figures/Pictures/APP1/Densiteter_GOOG.pdf}. Furthermore, they both resembles a bell-shaped density with mean approximately ATM, where the left plot in \autoref{Figures:Figures/Pictures/APP1/dens_sigmoid_GOOG.pdf} still has a drop below zero on the left side of the density, just as the one in the left plot in  \autoref{Figures:Figures/Pictures/APP1/Densiteter_GOOG.pdf}. These densities integrate to $0.76$ and $0.86$, respectively, and hence a bit closer and a bit further form one compared with the ones calculated using ReLU as the activation function. This could be caused by the fact that the sigmoid function gives a worse approximation of the implied volatility i general, but it is both one and two times differentiable and hence is smoother. 

As seen, there are both advantages and disadvantages of using ReLU and sigmoid as the activation function in the neural network. This will be further investigated in the following section, where the densities, calculated from the neural networks using either ReLU or sigmoid, will be used to calculate the option prices.  


\section{Calculating Option Prices}\label{Sec.App:Calculating_Option_Prices}
Given that we are able to determine the risk-neutral density, for all time to maturities, we should according to \eqref{Eq:RND_Pricing} be able to determine option prices given the current stock price $S_t=100$. Here, we will start with a detailed presentation of the results when using the ReLU as activation function, follow by a short summary of those achieved when using the sigmoid function. 

When looking at the densities in the former section it could be expected that the results improve when being further from maturity time when using ReLU as activation function in the neural network. To test these expectations, we sample $100$ random data points from the test set with varying time to maturities, $\tau$, and strike prices, $K$. Specifically, the distribution of $K$ and $\tau$ for each of the $100$ points can be seen in \autoref{Figures:Figures/Pictures/APP1/KT_GOOG.pdf}. The indices in the figures thus refers to point $1$ to $100$.

\imgfig[1]{Figures/Pictures/APP1/KT_GOOG.pdf}{Distribution of the strike prices and time to maturities for the $100$ random points in the test set.}

As seen in \autoref{Figures:Figures/Pictures/APP1/KT_GOOG.pdf}, the $100$ random points fairly well represent the distribution of the data in general seen in \autoref{Figures:Figures/Pictures/KDist.pdf} and \autoref{Figures:Figures/Pictures/TDist.pdf}. Hence, it is also possible to analyse whether the strike price and time to maturity affect the results. 

The calculated prices (red) are presented in the left plot in \autoref{Figures:Figures/Pictures/APP1/realcalc_GOOG.pdf}, together with the real option prices (black). Furthermore, the error between the two is illustrated in the right plot, where the error is the percentage error. 

\imgfig[1]{Figures/Pictures/APP1/realcalc_GOOG.pdf}{Real- and calculated prices (left) and the percentage error (right), when using the ReLU function.}

In the plot on the left hand side of \autoref{Figures:Figures/Pictures/APP1/realcalc_GOOG.pdf} we observe that the calculated option prices are in general smaller than the real ones. Additionally, we observe that the calculated option prices do follow the real option prices. Moreover, in the plot on the right hand side of \autoref{Figures:Figures/Pictures/APP1/realcalc_GOOG.pdf} we observe that there are some major spikes in the percentage error. When investigating these points, they all have one thing in common, namely a strike price far OTM, which could indicate that the approximation of the densities have a tendency to have missing mass in its tails. This missing mass results in lower option prices. As opposed to what we expected after analysing \autoref{Figures:Figures/Pictures/APP1/Densiteter_GOOG.pdf}, the options which had strike prices below ATM performed well. This could indicate that not all densities close to maturity have the problem with missing mass below ATM as seen in \autoref{Figures:Figures/Pictures/APP1/Densiteter_GOOG.pdf} (left). The mean error of the calculated prices are approximately $2.18$. This does not seem as much, however, since the majority of the prices are between zero and five, this results in an percentage error of $305.67\%$. Hence, this indicates that this method does not predict the prices very well, which may be a consequence of using several numerical methods to determine the prices. We will return to this in the discussion, after we try using this method on several different underlying stocks, to see whether the results repeat themselves. 

To further examine whether the percentage error depends on either strike prices or time to maturity we chose to plot the percentage error with respect to the these, which can be seen in \autoref{Figures:Figures/Pictures/APP1/KT_fejl_GOOG.pdf}. 

\imgfig[1]{Figures/Pictures/APP1/KT_fejl_GOOG.pdf}{The percentage error with respect to strike price (left) and time to maturity (right).}

In the plot on the left hand side of \autoref{Figures:Figures/Pictures/APP1/KT_fejl_GOOG.pdf} we observe that there seems to be a clear relation between the percentage error and the strike price. Specifically, we observe that as the strike price increases beyond ATM the percentage error also increases. As already mentioned, this could be caused by the tails of the densities missing mass, resulting in lower and sometimes negative prices. Furthermore, option prices for higher strike prices are in general also smaller than those with small strike prices, resulting in the errors' magnitude being larger, which might explain what is seen in the percentage error. In the plot on the right hand side of \autoref{Figures:Figures/Pictures/APP1/KT_fejl_GOOG.pdf} we observe that there seems to be no clear relation between the time to maturity and the percentage error.

Having conducted the preceding analysis when using the ReLU function as the activation function in the neural network, we make an identical analysis when using the sigmoid function as the activation function. That is, we sample the same $100$ random points and calculate the option prices using the risk-neural density, and then compare these with the real prices. We can from this conclude that the mean percentage error of the option prices were $498.42\%$ when using the sigmoid function, which is approximately $200\%$ worse than when using ReLU. The real and calculated prices for the $100$ points can be seen in \autoref{Figures:Figures/Pictures/APP1/sigmoid_price_GOOG.pdf}, together with the percentage error between the two. 

\imgfig[1]{Figures/Pictures/APP1/sigmoid_price_GOOG.pdf}{Real- and calculated prices (left) and the percentage error (right) when using the sigmoid function.}

In \autoref{Figures:Figures/Pictures/APP1/sigmoid_price_GOOG.pdf} it is clearly seen that the errors are caused by many negative prices, which is probably caused by missing mass in the tails of the densities. This was probably also the case when using the ReLU activation function, however, when using the sigmoid function this is more prominent when calculating the option prices. Here it is seen that the choice of activation function can make a big difference and perhaps many of the hyperparameters can also make a big difference for the final results. However, not being able to test all possible networks we continue with the one chosen in \autoref{Sec.App:NN} where ReLU is the activation function. This is done despite the fact that ReLU is not differentiable, since, for GOOG, this gave the best results when comparing it to the use of sigmoid. 


\section{Simulation Study}
To test our codes capabilities of calculating the correct risk-neutral distribution we create a simulation study. In this simulation study we want to simulate the distribution of $S_T\mid S_t$ using the \emph{stochastic volatility} Heston model. The Heston model describes how the stock price $S_t$ evolves over time depending on numerous variables, such as the stochastic volatility $V_t$. Thus, under the risk-neutral measure $\Q$ we are, on matrix form, working with the \emph{stochastic differential equations}
\begin{align}\label{Eq:EM}
    \mathrm{d}\begin{bmatrix}S_t\\V_t\end{bmatrix} &= \begin{bmatrix}rS_t\\\alpha-\lambda V_t\end{bmatrix}\,\mathrm{d}t + \begin{bmatrix}S_t\sqrt{1-\rho^2}&S_t\rho\\0&\sigma_v\sqrt{V_t}\end{bmatrix}\,\mathrm{d}\begin{bmatrix}W^{(1)}_t\\W^{(2)}_t\end{bmatrix},
\end{align}
where $W^{(1)}$ and $W^{(2)}$ are Wiener processes. Furthermore, $\lambda$ is the rate at which $V_t$ reverts to the long variance $\alpha/\lambda$, $\rho$ the correlation between the Wiener processes, and $\sigma_v$ the volatility of the volatility. To simulate these stochastic differential equations we use the Euler-Maruyama method, \citep{Gatheral}. The Euler-Maruyama method numerically approximates \eqref{Eq:EM} by discretising to a partition $t=t_0<\dots<t_n=T$ for $n\in\N$. This partition is then used to define \eqref{Eq:EM} as a recursive equation given as
\begin{align*}
        \mathrm{d}\begin{bmatrix}S_{t_k}\\V_{t_k}\end{bmatrix} &= \begin{bmatrix}rS_{t_k}\\\alpha-\lambda V_{t_k}\end{bmatrix}\left(t_{k+1}-t_k\right) + \begin{bmatrix}S_{t_k}\sqrt{1-\rho^2}&S_{t_k}\rho\\0&\sigma_v\sqrt{V_{t_k}}\end{bmatrix}\begin{bmatrix}(W^{(1)}_{t_{k+1}}-W^{(1)}_{t_k})\\(W^{(2)}_{t_{k+1}}-W^{(2)}_{t_k})\end{bmatrix},\quad 0\leq k\leq n-1.
\end{align*}
For the purpose of our simulation study we set $S_t=100$, $V_t=0.2$, $\alpha=0.5$, $\lambda=0.5$, $\rho=0.7$, and $\sigma_v=0.1$ when generating prices. Being able to simulate $S_T\mid S_t$ we use Monte Carlo simulation \citep[p. 267]{Hull} to get enough data for it to be representative of the risk-neutral distribution. This process is repeated for each chosen time to maturity, $\tau=T-t$. 

Additionally, these stock prices are used to calculate the option price using the formula
\begin{align}
    C(t,K) &= \exp(-r\tau)\E[\Q]{(S_T-K)^+\mid S_t}.
\end{align}
In practice, we simulate $100$ million values of $S_T\mid S_t$ and determine the mean of $\max(S_T-K,0)$ for all of these stock prices. We calculate option prices for values of $K$ and $\tau$ in $(0,250]$ and $(0,0.1]$, respectively. Given these option prices, we use the formula \eqref{Eq:IV_formula} to determined the implied volatility of the option, similar to before. All of this, meaning strike price, time to maturity, and implied volatility is then used to construct a neural network. The architecture of this neural network is chosen to be the same as in \autoref{Sec.App:Calculating_Option_Prices}, with strike price and time to maturity as inputs and implied volatility as output. This neural network gives us the implied volatility surface in \autoref{Figures:Figures/Pictures/SynVolSurf.png}. Compared to \autoref{Figures:Figures/Pictures/Application/vol_surfaceNN.pdf}, this neural network gives a very clear volatility skew (see \autoref{Sec.Implied_Volatility}) close to maturity time. Furthermore, we observe that the implied volatility for the options OTM vary more than those ITM.

\imgfig[0.7]{Figures/Pictures/SynVolSurf.png}{Implied volatility surface generated from simulated data for $S_t=100$, strike price $K$, and time to maturity, $\tau=T-t$.}

Given the approximation of the function for implied volatility we do as in \autoref{Sec.App:RND} and determine $\frac{\partial^2}{\partial K^2}C(t,K)$. As in \autoref{Sec.App:RND} we do so using the finite difference method to determine the derivatives for the implied volatility, and \eqref{Eq:RND_Derivatives}. Having determined all of the expressions in \eqref{Eq:RND_Derivatives}, we use $\frac{\partial^2}{\partial K^2}C(t,K)$ to determine the risk-neutral density using \eqref{Eq:App_RND}.

For $\tau=0.07$ this yields the density on the left hand side of \autoref{Figures:Figures/Pictures/SynDens.pdf}. Comparing this density to the one given by Monte Carlo simulation (see right hand side of \autoref{Figures:Figures/Pictures/SynDens.pdf}), we observe that it is not as leptokurtic in comparison. It does, however, as we expect lie very close to $K=100$. Additionally, it integrates to $1$ despite its look, making this part of the result better than the one achieved for GOOG (see \autoref{Figures:Figures/Pictures/APP1/Densiteter_GOOG.pdf}). Furthermore, the one for GOOG is not as smooth, which could be due to data restraints or numerical errors.

\imgfig[0.9]{Figures/Pictures/SynDens.pdf}{Actual risk-neutral density simulated using Monte Carlo and Euler-Maruyama (right), and calculated risk-neutral density using method described in \autoref{Sec.Calibration} (right).}

Thus, the approach we also use for the stocks, seems to work very well, even though the implied volatility points are not smooth. As opposed to the densities calculated for the stocks, we will not use the density from the left hand side of \autoref{Figures:Figures/Pictures/SynDens.pdf} to determine option prices, as the purpose of this simulation study was to determine its effectiveness in estimating the risk-neutral density.





\section{Comparison of stocks}
As mentioned in the beginning of the application, this project is not solely focused on GOOG, but also AAPL, AMZN and TSLA. We filter the data for stock prices, as done with GOOG, such that we have a clear current stock price. For AAPL, AMZN and TSLA their current stock price is $165$, $103$ and $161$, respectively. Additionally, the data was filtered such that we are only working with time to maturities smaller or equal to $0.1$. The big difference between these underlying stocks and GOOG is that there are approximately twice as many data points, more precisely between $24.000$ and $26.000$. This also results in more training data, which might improve the training process of the neural network and hence also the accuracy of the network. However, we will not be performing the same tuning process as for GOOG, but we will implement this data into a neural network with the same architecture determined throughout \autoref{Sec.App:NN}. This could result in the opposite of more training data, that is, a worse performance because the hyperparameters are not tuned for these data sets.      

Having filtered the data and calculated their respective implied volatilities, the implied volatility for AAPL, AMZN and TSLA is plotted in \autoref{Figures:Figures/Pictures/TAA/vol_AAPL.png}, \autoref{Figures:Figures/Pictures/TAA/vol_AMZN.png}, and \autoref{Figures:Figures/Pictures/TAA/vol_TSLA.png}, respectively. Comparing these to the one for GOOG, in \autoref{Figures:Figures/Pictures/Volatility Surface Scatter (GOOG).png}, we firstly observe that all of them have a volatility smile present close to maturity. Secondly, we observe that AMZN and TSLA are in general more level compared to both GOOG and AAPL further from maturity. Lastly, we observe for AMZN that it seems to have two parallel surfaces, one more than we were expecting. This could be caused by some numerical problems when calculating the implied volatility using \eqref{Eq:IV_formula}. When assuming stationarity of the risk-neutral density, we also assume that the implied volatility is stationary. What we see in \autoref{Figures:Figures/Pictures/TAA/vol_AMZN.png} can be an indication that this assumption does not hold for AMZN. However, we will not do anything to smoothen this surface, to see how a volatility surface which behaves very different affects the results. 

\timgfig{Figures/Pictures/TAA/vol_AAPL.png}{Implied volatility of AAPL for $S_t=165$.}{Figures/Pictures/TAA/vol_AMZN.png}{Implied volatility of AMZN for $S_t=103$.}{Figures/Pictures/TAA/vol_TSLA.png}{Implied volatility of TSLA for $S_t=161$.}

As we did for GOOG, we first look at how the strike prices and time to maturities are distributed for each of the stocks, to get an indication of what the data is being trained on. The distribution of these variables for AAPL, AMZN and TSLA can be seen in \autoref{Figures:Figures/Pictures/TAA/hist_AAPL.pdf}, \autoref{Figures:Figures/Pictures/TAA/hist_AMZN.pdf}, \autoref{Figures:Figures/Pictures/TAA/hist_TSLA.pdf}, respectively. Firstly, for AAPL we observe that the majority of strike prices are between $160$ and $170$, whilst the time to  maturities are fairly evenly spread out. Secondly, for AMZN we observe that the majority of strike prices are between $100$ and $110$, with the time to maturities being further from maturity time. Lastly, for TSLA we observe that the strike prices are between $160$ and $180$, whilst the time to maturities are fairly spread out, with a bit more points close to maturity time.

We then train a neural network with the architecture presented in \autoref{Sec.App:NN}, with the ReLU function as the activation function. Training a neural network with the same architecture for each of the stocks, presents the risk of both over- and underfitting. Hence, we look at how the MSE, MAE and MAPE evolve, for each of the stocks, during the training process of $1200$ epochs. These can be seen in \autoref{Figures:Figures/Pictures/TAA/fejl_aapl.pdf}, \autoref{Figures:Figures/Pictures/TAA/fejl_amzn.pdf}, and \autoref{Figures:Figures/Pictures/TAA/fejl_tsla.pdf} for AAPL, AMZN, and TSLA, respectively. For all of them there seems to be very little, if any, overfitting, which is desired. However, in all of them we see some variation in the error of the validation set, as we did for GOOG. 

Moreover, we evaluate each network by calculating the MSE, MAE and MAPE for each of them on the test sets. The values can be seen in \autoref{Tab:APP:MSE MAE MAPE}, with the takeaway being that AAPL has a better general performance than both GOOG, AMZN and TSLA. Likewise, TSLA's performance is not much worse than GOOG's performance, with it only having a MAPE $0.7\%$ higher. On the opposite end of the spectrum is AMZN's performance, which is not as good as the others. This could be caused by the implied volatility surface in \autoref{Figures:Figures/Pictures/TAA/vol_AMZN.png} not being as smooth as the others, making it difficult for the neural network to approximate it.

\begin{table}[H]
    \centering
    \subcaptionbox{AAPL}{
        \begin{tabular}{c|c}
            MSE  &  0.8565e-5\\ \hline
            MAE  &  5.2188e-3\\ \hline
            MAPE &  1.7590\%\\ 
        \end{tabular}
    }
    \hfill
    \subcaptionbox{AMZN}{
        \begin{tabular}{c|c}
            MSE  &  0.0015\\ \hline
            MAE  &  0.0259\\ \hline
            MAPE &  6.3934\%\\ 
        \end{tabular}
    }
    \hfill
    \subcaptionbox{TSLA}{
        \begin{tabular}{c|c}
            MSE  &  0.0006\\ \hline
            MAE  &  0.01624\\ \hline
            MAPE &  2.8651\%\\ 
        \end{tabular}
    }
    \hfill
    \caption{MSE, MAE and MAPE the the test set of AAPL, AMZN and TSLA.}
    \label{Tab:APP:MSE MAE MAPE}
\end{table}

The implied volatility surface created by these neural networks are illustrated in \autoref{Figures:Volatility_Surfaces} for AAPL, AMZN and TSLA, respectively. For both AAPL and TSLA their surfaces are relatively smooth, with AAPL's being more symmetric around ATM whilst TSLA's is more skewed towards ITM. On the contrary, the implied volatility surface for AMZN reflects what we saw in \autoref{Figures:Figures/Pictures/TAA/vol_AMZN.png}, namely that it tries to approximate the two implied volatility surfaces which results in a surface with some dips, instead of a smoother surface as for the two other stocks. The reason why it is actually possible for the network to model the data as well as it does, could be because the training data set is quite large and it hence has the possibility to replicate this behaviour, or it could be caused by the ratio between parameters and data. This will be discussed further at a later point in the project. 

\timgfigto{Figures/Pictures/TAA/volsurf_AAPL.png}{Figures/Pictures/TAA/volsurf_AMZN.png}{Figures/Pictures/TAA/volsurf_TSLA.png}{Implied volatility surface of respectively AAPL, AMZN, and TSLA from the neural networks compared with their test sets (orange points).}{Volatility_Surfaces}

Wanting to test how these neural networks performed on their respective test sets, we used prediction plots, to determine their capabilities. The prediction plots for AAPL, AMZN and TSLA can be seen in \autoref{Figures:Figures/Pictures/TAA/pred_real_aapl.pdf}, \autoref{Figures:Figures/Pictures/TAA/pred_real_amzn.pdf}, and \autoref{Figures:Figures/Pictures/TAA/pred_real_tsla.pdf}, respectively. As with GOOG, both the theoretical best regression and the actual linear regression lines are illustrated in black and blue, respectively. To empirically summarise their capabilities, the linear regressions $R^2$ values were calculated, these being: $0.97$, $0.85$, and $0.97$ for AAPL, AMZN and TSLA, respectively. Again, what this indicates is that AMZN's performance is poor compared to the others, whilst AAPL and TSLA actually perform better than GOOG (see \autoref{Figures:Figures/Pictures/Application/lineplot.pdf} and \autoref{tab:summary_af_lm}).

Given a neural network for each of the stocks, we move on to determine their risk-neutral densities. For AAPL, AMZN and TSLA we choose to look at the same two time to maturities as for GOOG, namely $\tau=0.02$ and $\tau = 0.07$, with their respective risk-neutral densities illustrated in \autoref{Figures:Figures/Pictures/APP1/Densiteter_AAPL.pdf}, \autoref{Figures:Figures/Pictures/APP1/Densiteter_AMZN.pdf}, and \autoref{Figures:Figures/Pictures/APP1/Densiteter_TSLA.pdf}. In all of these densities, a distinct pattern can be observed, similar to what was observed for GOOG. Specifically, as the stock price at maturity deviates further from the current stock price, the densities becomes increasingly more distorted or fluctuates more significantly. Additionally, for the densities associated with AMZN, even ATM, there is a slight degree of instability. This could be explained by the error of the network for the implied volatility for AMZN in general being larger than for the other stocks. Lastly, when integrating the densities for the three stocks we get that they all integrates to something smaller than one. For AAPL and the two time to maturities they integrate to $0.78$ and $0.76$, respectively, for AMZN it is $0.82$ and $0.665$, and for TSLA it is $0.54$ and $0.647$. Furthermore, from this very small sample it indicates that TSLA is performing the worst, which is unexpected since AMZN, in general, performed worse in every metric when looking at the neural networks.

To test how well these risk-neutral densities perform in general, we choose $100$ random points from the stocks' respective test sets and compare the real prices to those calculated using the risk-neutral densities. These points and their distribution for each stock can be seen in \autoref{Figures:Figures/Pictures/APP1/KT_AAPL.pdf}, \autoref{Figures:Figures/Pictures/APP1/KT_AMZN.pdf} and \autoref{Figures:Figures/Pictures/APP1/KT_TSLA.pdf} with the calculated prices and their percentage errors seen in \autoref{Figures:Figures/Pictures/APP1/realcalc_AAPL.pdf}, \autoref{Figures:Figures/Pictures/APP1/realcalc_AMZN.pdf} and \autoref{Figures:Figures/Pictures/APP1/realcalc_TSLA.pdf}. 

\vspace{-1.5em}\imgfig[1]{Figures/Pictures/APP1/realcalc_AAPL.pdf}{Real- and calculated prices (left) and the percentage error (right) for AAPL.}\vspace{-2.5em}
\imgfig[1]{Figures/Pictures/APP1/realcalc_AMZN.pdf}{Real- and calculated prices (left) and the percentage error (right) for AMZN.}\vspace{-2.5em}
\imgfig[1]{Figures/Pictures/APP1/realcalc_TSLA.pdf}{Real- and calculated prices (left) and the percentage error (right) for TSLA.}\vspace{-0.5em}

As we saw for GOOG the calculated
option prices follow the real option prices, which can be seen in the left hand side of the three figures above. However, when looking at their percentage error we observe that it is still large for all of the stocks. The biggest difference between the stocks is their errors magnitude, with AMZN's being extremely large compared to the others. For a quick summary of the results achieved from these $100$ points, see \autoref{Tab:App:OP_Errors}.

\begin{table}[H]
     \centering
     \begin{tabular}{c|cc}
               & Mean Error & Mean Percentage Error \\ \hline
          AAPL & $1.1809$ & $103.4964\%$ \\
          AMZN & $1.5657$ & $902.5436\%$ \\
          TSLA & $1.1685$ & $99.4070\%$
     \end{tabular}
     \caption{Mean error and mean percentage error between real and calculated option prices for AAPL, AMZN, aand TSLA for 100 random point from their respective test sets.}
     \label{Tab:App:OP_Errors}
 \end{table}

Looking at the values in \autoref{Tab:App:OP_Errors} we clearly see the same behaviour as for GOOG, that is, the mean errors being fairly small but having an enormous mean percentage error. This is again explained by the fact that most prices are below five, and hence an even small error is a large percentage error. Furthermore, when looking at \autoref{Figures:Figures/Pictures/APP1/realcalc_AAPL.pdf}, \autoref{Figures:Figures/Pictures/APP1/realcalc_AMZN.pdf} and \autoref{Figures:Figures/Pictures/APP1/realcalc_TSLA.pdf}, it is clear that the option prices which are calculated using the risk-neutral density are always smaller than the real ones. This was to be expected after analysing the risk-neutral densities for the stocks earlier in the section, and comparing them to GOOG. Moreover, when looking at the stocks' percentage error with respect to their strike prices and time to maturities we observed similar behaviour as for GOOG. Specifically, we observed that as the strike price increases so does the percentage error, whereas there was no clear relation between time to maturity and percentage error.

The two examples of risk-neutral densities for the three stocks indicated that TSLA in general would perform the worst, nevertheless we see that it outperforms not only AAPL and AMZN, but also GOOG which had a mean percentage error of approximately $300\%$. The results for GOOG are not only outperformed by TSLA but also AAPL by approximately $200\%$, where in both cases this could be caused by a general better performance of the neural network potentially due to more data. Lastly, the noticeably bad performance for AMZN could again be caused by the not smooth implied volatility surface, which not only can affect the values for implied volatility but also the derivatives. 

In general, we see that this method of predicting the option prices is fairly unstable, among other things, because of small option prices. Again, this could be explained by the implied volatility surface, both the one constructed by the data and the neural network, not being smooth. This will also affect the derivatives of the implied volatility which are used to construct the risk-neutral density, and hence affecting the final result. Furthermore, we have seen by comparing the results for GOOG and AMZN that just because there are more data for the neural network to be trained upon the results will not necessarily improve. This could also be caused by the neural network not being tuned for AMZN but GOOG. However, comparing GOOG, AAPL and TSLA there is an indication of an improvement when having more data. Hence, the worse results for AMZN are probably caused by something else. This divergence can be attributed to the substantial dissimilarities in the data from these stocks, particularly in terms of the range of strike prices available. Based on our observations, the approach of determining option prices using the risk-neutral density yielded the best results for TSLA, followed by AAPL and GOOG, while it performed the worst for AMZN. This coincides with the available strike prices for TSLA being between $20$ and $420$, whilst for AAPL, GOOG, and AMZN they are between $70$ and $280$, $50$ and $180$, and $50$ and $160$, respectively. Hence, the training set for the neural network for TSLA extends further away from the ATM. Consequently, the network is expected to perform better in predicting values further from ATM compared to the neural networks for the other stocks. This, in turn, affects the risk-neutral density, leading to an improvement in the tails and thereby better predictions of the option prices. Lastly, we conclude that this method still gives an error of at least $100$ percent or more, which is not optimal. This could be caused by using multiple numerical approximation methods to get the final result, such as the neural network, the derivatives of this and the integral in \eqref{Eq:RND_Pricing}. The poor results of the predictions could also be caused by the assumption of the risk-neutral density being stationary. All of this will be further discussed in \autoref{Ch.Discussion}. To address the problem of using multiple numerical approximation methods back-to-back we will try to use neural networks to approximate the option price function directly instead of the implied volatility function. 


 