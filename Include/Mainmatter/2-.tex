\chapter{Theory}\label{Ch.2}
This chapter is based upon \citep{Hull} and \citep{Cal}, unless stated otherwise.

Given a financial market in which many derivatives are traded a common financial problem is pricing these. Of course the price must be $\exp\left(-rT\right)\E[\Q]{\Phi(s)}$, given that $\Q$ is a \emph{risk neutral measure}, $r>0$ is the risk-free rate and $\Phi(s)$ is the pay-off function given that the price is $s$. For markets such as the Black-Scholes market where the volatility is constant, the price is risk-neutral etc. \citep{BS}. This problem is not as difficult as in the markets where these assumptions do not hold. Specifically most financial markets tend to be incomplete \textbf{(Kilde)} hence multiple prices exist. Before addressing this drawback further the project focuses on using options to determine the \emph{implied volatility}.


\section{Implied Volatility}\label{Sec.Implied_Volatility}
Implied volatility is often interpreted as the market's prediction of future volatility and similarly the evolution of the implied volatility reflects how the market responds to new information. This does not mean that implied volatility can be used to predict whether or not the price will increase or decrease in the future. Furthermore, remember that implied- and historical volatility are not identical; historical- describes the past, whilst implied volatility is interpreted as the future.

Even though determining implied volatility is a central problem in finance, there is not a universally accepted method of determining it. A rather used method is, however, to solve
\begin{align}\label{Eq:IV_formula}
    C^{M}(T,K) &= C^{BS}\left(T,K,\IV{T,K}\right),\quad T,K>0,
\end{align}
where $C^{M}$ is the market's call option price, $C^{BS}$ is the Black-Scholes call option price on the same underlying asset and $\IV{T,K}>0$ is the implied volatility. In practice, numerical methods such as the Newton Raphson or a similar decent method is used to determine the implied volatility such that it satisfies \eqref{Eq:IV_formula}.


\subsection{Volatility Surface}\label{SSec.Volatility_Surface}
As an investor you might feel indifferent about knowing the implied volatility of a single option as it tells not much, if anything, about the underlying stock. So, to give a better understanding of the volatility of the options on an underlying stock investors look at its \emph{implied volatility surface}. 
\imgfig[0.25]{Figures/Pictures/Volatility Surface.pdf}{Volatility surface for Moneyness: $K/S$ and time to maturity: $T-t$.}
As seen in \autoref{Figures:Figures/Pictures/Volatility Surface.pdf} the closer the options get to maturity, the closer it resembles the \emph{volatility smile}. Similarly, the further the option is to maturity, the more likely it is to be close to 0. 


\section{Calibration}\label{Sec.Calibration}
As mentioned in the beginning of the chapter it is a drawback that financial markets tend to be incomplete. This has the implication, that given a derivative's market price can be described by $\pazocal{P}\left(\theta,\alpha\right)$ where $\alpha$ is the derivatives fixed parameters (e.g. maturity time $T$ and interest rate $r$) and $\theta$ is the model parameters, then infinitely many prices exist. Given that the model is accurate, then there exists a $\theta^m$ such that $\pazocal{P}\left(\theta^m,\alpha\right)$ is the market's price for a given $\alpha$. Thus an agent would like to estimate $\theta^m$ to determine the market's price. This approach is what is known as \emph{model calibration} as it utilises the pricing model. Once more, given a closed form model such as the one in the Black-Scholes market, this task is trivially solved, whereas for other, non-closed form, models $\theta^m$ has to solved numerically. 

Another approach, which does not depend on a model, is \emph{risk-neutral calibration}, which attains its name from utilising the \emph{risk neutral density function}, $f_T^\Q$. Thus, given that one is working with \emph{European options} the general pricing formula, presented in the beginning is equal to
\begin{align}\label{Eq:RND_Pricing}
    \exp\left(-rT\right)\E[\Q]{\Phi(S_T)} &= \exp\left(-rT\right)\int_\R\Phi(s)f_T^\Q(s)\,\mathrm{d}s.
\end{align}

For a European call option with price $C$ and pay-off function $\Phi(S_T)=\max(0, S_T-K)$ for $S_T, K>0$, then \eqref{Eq:RND_Pricing} is equivalent to
\begin{align}\label{Eq:RND_Calibration}
    C(T,K) &= \exp\left(-rT\right)\int_K^\infty (s-K)f_T^\Q(s)\,\mathrm{d}s.
\end{align}
It should be intuitively clear that if $C$ depends on a given set of parameters $\theta$ then $C(T,K)$ could differ, depending on $\theta$. As mentioned previously, the goal of calibration thus becomes determining $f_T^{\Q}$ as it is the only unknown in \eqref{Eq:RND_Calibration}. 

Using \eqref{Eq:RND_Calibration} one obtains that conditioned on $K=s$, then
\begin{align}\label{Eq:PartialC}
    \frac{\partial^2}{\partial K^2}C(T,K) &= \exp\left(-rT\right)f_T^\Q(s),
\end{align}
where $\exp\left(-rT\right)$ is deterministic hence $\frac{\partial^2}{\partial K^2}C(T,K)$ describes $f_T^\Q(s)$. Thus, to determine $f_T^\Q(s)$ one just has to determine $\frac{\partial^2}{\partial K^2}C(T,K)$. 


\textbf{Antag at formlen er givet korrekt}\\
Using \textbf{TBD}, \eqref{Eq:PartialC} can be written as
\begin{align}\label{Eq:RND_Derivatives}
    \frac{\partial^2}{\partial K^2}C(T,K) &= \frac{\partial^2}{\partial K^2}C^{BS}\left(T,K,\IV{T,K}\right)+2\frac{\partial^2}{\partial K\partial\sigma}C^{BS}\left(T,K,\IV{T,K}\right)\frac{\partial}{\partial K}\IV{T,K}\\
    &+\frac{\partial^2}{\partial\sigma^2}C^{BS}\left(T,K,\IV{T,K}\right)\left(\frac{\partial}{\partial K}\IV{T,K}\right)^2+\frac{\partial}{\partial\sigma}C^{BS}\left(T,K,\IV{T,K}\right)\frac{\partial^2}{\partial K^2}\IV{T,K},
\end{align}
which in part tells that one only has to determine $\IV{T,K}$ as well as its derivatives to determine $f_T^\Q(s)$. Given that implied volatility, $\IV{T,K}$, does not have a closed form solution it has to be approximated numerically. A fairly well-known method of estimating derivatives is the finite difference method presented in the following section. Moreover, the project will focus on how a neural network can be constructed to estimate the implied volatility (See \autoref{Ch.3}).


\section{Finite Difference}\label{Sec.Finite_Difference}
The principal of the \emph{finite difference} method is to numerically solve ordinary differential equations using difference quotients \citep{Hull}. The finite difference method has multiple submethods, such as \emph{forward difference} and \emph{backward difference} which for a function $f$ are given by
\begin{align*}
    \frac{\partial f}{\partial t}=\frac{f(t+\Delta)-f(t)}{\Delta}+\pazocal{O}(\Delta)\quad\textrm{and}\quad\frac{\partial f}{\partial t}=\frac{f(t)-f(t-\Delta)}{\Delta}+\pazocal{O}(\Delta),
\end{align*}
respectively. It should be trivial to see that $\pazocal{O}(\Delta)$ decreases as $\Delta$ decreases, minimising the error of the derivative. The finite difference method also has a submethod called the \emph{central difference} which for a function $f$ is given by
\begin{align*}
    \frac{\partial f}{\partial t}=\frac{f(t+\Delta)-f(t-\Delta)}{2\Delta}+\pazocal{O}(\Delta^2)
\end{align*}
where the error $\pazocal{O}(\Delta^2)$ is smaller than that of the forward- and backward difference. All of these methods can be repeated $k$ times if one has to determine the $k$ derivative of a function $f$.

If $\Delta$ is small and $\partial f/\partial t$ is continuous, then it should be a good approximation. Thus given the a neural network or another function estimating method produces a smooth enough function, finite difference can be easily implemented, especially for lower order derivatives, meaning the derivatives of $\mathrm{IV}$ in \eqref{Eq:RND_Derivatives} can be determined numerically. All this talk about neural networks leads one to look closer at the subject in \autoref{Ch.3}.




\nomenclature[]{$C^M$}{Market's call option price}
\nomenclature[]{$C^{BS}$}{Black Scholes call option price}
\nomenclature[]{$\IV{\cdot}$}{Implied Volatility}