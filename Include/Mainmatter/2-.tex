\chapter{Preliminary Theory}\label{Ch.2}
This chapter is based upon \citep{Hull} and \citep{Cal}, unless stated otherwise.

Given a financial market in which many derivatives are traded, a common financial problem is, as mentioned, pricing these. A general formula for the risk-neutral price of a simple European derivative is $\exp\left(-r\tau\right)\E[\Q]{\Phi(S_T)\mid \mathcal{F}_t}$ where $\Q$ is a \emph{risk-neutral measure}, $r\in\R$ is the risk-free interest rate, $\tau=T-t$ is the time to maturity, $\Phi(S_T)$ is the pay-off function, and $S_T$ is the price of the underlying stock at maturity time, $T$. Additionally, $\mathcal{F}_t$ is the sub-$\sigma$-algebra of $\mathcal{F}$\footnote{The $\sigma$-algebra in the filtered probability space $\left(\Omega,\mathcal{F},\left(\mathcal{F}_t\right)_{t\in[0,T]},\mathbb{P}\right)$} for which $\mathcal{F}_s\subseteq\mathcal{F}_t\subseteq\mathcal{F}$ given $s\leq t\leq T$. The intuition of this sub-$\sigma$-algebra, $\mathcal{F}_t$, is that it represents the information in the market up until a time $t$, hence why it is increasing as time passes.

Given that an investor knows the formula, the problem of pricing derivatives is not trivial, as investors are required to know many complex market factors such as global economic politics, supply and demand, market sentiment, and \emph{implied volatility}. Hence, a model which depends on these complex market factors is often created such that it coincides with the market prices. These models can also vary depending on which kind of derivative one is working with. As options are often used to manage risk and speculate on future price movements these will be the focus going forward. This project will only focus on simple \emph{European call options}, as the put-call parity can be used to relay the findings. Remember that the pay-off function for a simple European call option is $\Phi(S_T) = (S_T - K)^+$ for strike price $K>0$. As mentioned, one of the complex market factors in option pricing is the implied volatility, why the following section will focus on its properties.


\section{Implied Volatility}\label{Sec.Implied_Volatility}
Implied volatility is commonly understood as the market's expectation or prediction of future volatility. Additionally, the changes in implied volatility are often seen as an indication of how the market reacts to new information. However, this does not mean that implied volatility can be used to predict whether or not the price will increase or decrease in the future. Investors can also use the implied volatility to risk manage their portfolio using the Greeks, which can be determined using the Black-Scholes formula, \citep{BS}. Heuristically what this mean is that an investors is able to measure the option prices sensitivity with respect to time (Theta), volatility (Vega), etc. also making it possible to hedge. Additionally, given an investor is able to determine the implied volatility, they are able to price options, meaning that for a strike $K$ and time to maturity $\tau=T-t$, they are able to determine the price using the Black-Scholes formula. This can also be inverted to calculate the implied volatility, given the price of an option. More precisely, one has to solve
\begin{align}\label{Eq:IV_formula}
    C^{M}(t,K) &= C^{BS}\left(t,K,\IV{t,K}\right),\quad \tau,K>0,
\end{align}
where $C^{M}\geq0$ is the market's call option price, $C^{BS}\geq0$ is the Black-Scholes call option price on the same underlying stock and $\IV{t,K}>0$ is the implied volatility. In practice, numerical methods such as the Newton Raphson or a similar decent method is used to determine the implied volatility such that it satisfies \eqref{Eq:IV_formula}.

As an investor you might not only be interested in knowing the value of the implied volatility of a single option. Thus, to give a better understanding of the implied volatility of the options on an underlying stock investors look at its \emph{implied volatility surface}. The volatility surface makes it possible for investors to get an idea of what the market expectations are, the supply and demand as well as the sentiment towards the underlying asset. However, for the prices determined from the implied volatility surface to be arbitrage-free, the surface has to be determined from arbitrage-free prices. If this is not the case, these prices might not be arbitrage-free. In \autoref{Figures:Figures/Pictures/Volatility Surface Smile.png} and \autoref{Figures:Figures/Pictures/Volatility Surface Skew.pdf} two examples of what an implied volatility surface might look like is presented. These two implied volatility surfaces are created from data with a current underlying stock price of $100$.

\dimgfig{Figures/Pictures/Volatility Surface Smile.png}{Implied volatility surface with current underlying stock price $S_t=100$, strike price, $K$, and time to maturity, $\tau=T-t$.}{Figures/Pictures/Volatility Surface Skew.pdf}{Implied volatility surface with current underlying stock price $S_t=100$, strike price, $K$, and time to maturity, $\tau=T-t$.}

It is evident in \autoref{Figures:Figures/Pictures/Volatility Surface Smile.png}, the closer the options get to maturity, the more symmetric around at the money (ATM) it becomes, closely resembling what is known as the \emph{volatility smile}. What this means heuristically is that the options ATM have lower implied volatility than those out the money (OTM) and in the money (ITM). In \autoref{Figures:Figures/Pictures/Volatility Surface Skew.pdf} something different occurs, namely that close to maturity it does not resemble the volatility smile but instead a \emph{volatility skew}. This volatility skew, as it is called, can be caused by numerous factors such as the supply and demand, price sensitivities to changes in the underlying asset's price etc. Focusing on the supply and demand factor, it would in \autoref{Figures:Figures/Pictures/Volatility Surface Skew.pdf} mean that the demand for options OTM is larger than the demand for options ITM.

Again, given that an investor is able to determine the markets implied volatility surface this can be used to determine prices, which can be achieved either by using a model, historical data, or both. For both determining the function for the implied volatility and the option price, \emph{calibration} can be used, which will be elaborated further upon in the following section.


\section{Calibration}\label{Sec.Calibration}
Calibration, in finance, is the process of estimating the dynamics under the risk-neutral measure. Calibration refers to the process of adjusting model parameters to modify the model's predictions to observed market data or known values, to ensure that the model accurately reflects real-world conditions. In finance, calibration is commonly used with financial models, such as option pricing models. Consider a financial market in which option prices can be described by a model $\pazocal{P}\left(\bm\theta,\bm\beta\right)$. In this model, $\bm\theta$ represents the model's $n$ unknown parameters and $\bm\beta$ represents the options fixed parameters such as maturity time and risk-free interest rate. Thus, given $\bm\beta$, there must exist an associated $\widehat{\bm\theta}$ such that $\pazocal{P}\left(\widehat{\bm\theta},\bm\beta\right)$ is exactly the market price. This also implies that given a market price and $\bm\beta$, one should be able to determine $\widehat{\bm\theta}$. This is what is known as \emph{model calibration} as it requires a model. Formally, the problem of determining the model parameters can be described as
\begin{align*}
    \widehat{\bm\theta} &= \arg\min_{\bm\theta\in\Theta}\pazocal{C}\left(\pazocal{P}\left(\bm\theta,\bm\beta\right),\pazocal{P}\left(\widehat{\bm\theta},\bm\beta\right)\right),
\end{align*}
where $\pazocal{C}$ is a suitable metric and $\Theta\subseteq\R^n$ represents the set for which the model's constraints are fulfilled. Naturally, this methodology is easier given a model which has a closed-form solution, such as the Black-Scholes model, whereas for other models, with no closed-form solutions, $\widehat{\bm\theta}$ has to be determined numerically.

Another calibration method, which does not require a model, but only market data is the \emph{risk-neutral calibration}, which has its foundation in the general pricing formula, presented in the beginning. Thus, given that one is working with simple European call options, and there exists a \emph{risk-neutral density}, then the pricing formula is equal to
\begin{align}\label{Eq:RND_Pricing}
    C(t,K) &= \exp\left(-r\tau\right)\E[\Q]{\Phi(S_T)\mid S_t} = \exp\left(-r\tau\right)\int_\R(S_T-K)^+f^\Q(S_T\mid S_t)\,\mathrm{d}S_T,
\end{align}
given $K,T>0$ and $T>t$. On the right hand side of \eqref{Eq:RND_Pricing}, the only unknown is the risk-neutral density function $f^\Q$, which represents the conditional distribution of $S_T$ given the price $S_t$. Hence, if one can determine this risk-neutral density function one is able to describe how option prices behave under the risk-neutral measure $\Q$. Differentiating the right hand side of \eqref{Eq:RND_Pricing} with respect to the strike price twice for $K=S_T$ yields
\begin{align}\begin{aligned}\label{Eq:PartialC}
    \frac{\partial^2}{\partial K^2}C(t,K) &= \exp\left(-r\tau\right)\frac{\partial}{\partial K}\int_K^\infty -f^\Q(S_T\mid S_t)\,\mathrm{d}S_T\\
    &=\exp\left(-r\tau\right)f^\Q(S_T\mid S_t).
\end{aligned}\end{align}
Hence, being able to determine $\frac{\partial^2}{\partial K^2}C(t,K)$, it is also possible to derive $f^\Q$. Another way of writing $\frac{\partial^2}{\partial K^2}C(t,K)$ can be derived using the multivariate chain rule and \eqref{Eq:IV_formula}. Thus,
\begin{align}\begin{split}\label{Eq:RND_Derivatives}
    \frac{\partial^2}{\partial K^2} C(t,K) &= \frac{\partial}{\partial K}\left(\frac{\partial}{\partial K} C^{BS}\left(t,K,\IV{t,K}\right) + \frac{\partial}{\partial\IV{t,K}} C^{BS}\left(t,K,\IV{t,K}\right)\frac{\partial}{\partial K}\IV{t,K}\right)\\
    &= \frac{\partial^2}{\partial K^2}C^{BS}\left(t,K,\IV{t,K}\right)+2\frac{\partial^2}{\partial K\partial\IV{t,K}}C^{BS}\left(t,K,\IV{t,K}\right)\frac{\partial}{\partial K}\IV{t,K}\\
    &\phantom{=} +\frac{\partial^2}{\partial\left(\IV{t,K}\right)^2}C^{BS}\left(t,K,\IV{t,K}\right)\left(\frac{\partial}{\partial K}\IV{t,K}\right)^2\\&\phantom{=} +\frac{\partial}{\partial\IV{t,K}}C^{BS}\left(t,K,\IV{t,K}\right)\frac{\partial^2}{\partial K^2}\IV{t,K}.
\end{split}\end{align}
Thus, one only has to determine $\IV{t,K}$ as well as its derivatives to determine $f^\Q$, since the Black-Scholes formula has a closed-form solution, and its derivatives are known. On the contrary, the implied volatility, $\IV{t,K}$, does not have a closed form solution meaning it has to be approximated numerically. In the project's application this will be done using \emph{neural networks}, whilst its derivatives will be determined using the \emph{finite difference} method, both elaborated further upon in \autoref{Ch.3} and \autoref{Ch.4}, respectively.

\chapter{Finite Difference}\label{Ch.4}
This chapter is based upon \citep{Hull}, unless stated otherwise.

As mentioned in \autoref{Ch.2}, the finite difference method will be used to determine the derivatives of the implied volatility seen in \eqref{Eq:RND_Derivatives}. The principle of the finite difference method is to numerically solve ordinary differential equations using difference quotients \citep{Hull}. Consider the Taylor series for a function $f\in C^{\infty}$
\begin{align}\label{eq:central_diff}
    f(a) &= \sum_{k=0}^\infty \frac{\partial^k f(t)}{\partial t^k}\frac{1}{k!}(a-t)^k, \nonumber %+ o\left((x-a)^{k}\right),
\intertext{then it can be rewriting in terms of the central difference operator, which yields}
    f(t+\Delta) - f(t-\Delta) &= \sum_{k=0}^\infty \frac{\partial^k f(t)}{\partial t^k}\frac{1}{k!}(\Delta)^k - \sum_{k=0}^\infty \frac{\partial^k f(t)}{\partial t^k}\frac{1}{k!}(-\Delta)^k \nonumber \\
    &=2\frac{\partial f(t)}{\partial t}\Delta + 2\frac{\partial^3 f(t)}{\partial t^3}\frac{1}{3!}(\Delta)^3 + 2\frac{\partial^5 f(t)}{\partial t^5}\frac{1}{5!}(\Delta)^5 + \cdots. \nonumber
\intertext{Isolating for the first order derivative, and determining the error for the higher order derivatives yield}
    \frac{\partial f}{\partial t} &= \frac{f(t+\Delta)-f(t-\Delta)}{2\Delta}+o(\Delta^2),
\end{align}
which is called the \emph{central difference} method. In \eqref{eq:central_diff}, the error term $o(\Delta^2)$ is, as mentioned, the collection of the higher order derivatives. Likewise, if one were to use the forward- or backward difference operator the results would be
\begin{align*}
    \frac{\partial f}{\partial t}=\frac{f(t+\Delta)-f(t)}{\Delta}+o(\Delta)\quad\textrm{and}\quad\frac{\partial f}{\partial t}=\frac{f(t)-f(t-\Delta)}{\Delta}+o(\Delta),
\end{align*}
named accordingly. As the project also concerns approximating second order derivatives, the second order central difference method is given as
\begin{align*}
    \frac{\partial^2 f}{\partial t^2} &= \frac{f(t+\Delta) - 2f(t) + f(t-\Delta)}{\Delta^2}+o(\Delta^2),
\end{align*}
where its derivation is similar to the first order central difference method.

In the equations above $o(\Delta)$, and $o(\Delta^2)$ decrease as $\Delta$ decreases, minimising the error of the derivative. Naturally, $o(\Delta^2)$ decreases the fastest for $|\Delta|<1$, why the central difference method is often used in cases where both methods are applicable. This is coherent with the methods, without their respective error terms, being good approximations if $\Delta$ is small.

To conclude, given a neural network or another function estimating method which produces a smooth enough function, finite difference can be easily implemented, especially for lower order derivatives. 


%\section{Partial Differential Equation}\label{Sec.Finite_Difference}
%The finite difference methods use cases can also be extended further than simply approximating derivatives; its methodology can be implemented to describe partial differential equations (PDEs). Given a solution to such a PDE, this solution can be approximated using the finite difference methods described beforehand.

%For the Black-Scholes equation, the derivatives are with respect to time $t$ and stock prices $S$. As such, for the finite difference method to be useful, one has to discretise the domain of these two variables on which the finite difference method will be applied. To visually aid this description see \autoref{Figure:Finite_Difference}.
%\begin{figure}
    \centering
    \begin{center}
    \usetikzlibrary {datavisualization.formats.functions}
    \tikz \datavisualization
      [scientific axes,
       all axes={
         length=7.5cm,
         grid,
       },
       grid layer/.style=, % none, so on top of data (bad idea)
       visualize as line]
      data [format=function] {
        var x : interval [0:10];
        func y = max(\value x-5, 0);
      };
      \end{center}
    \caption{Finite difference of Black-Scholes.}
    \label{Figure:Finite_Difference}
\end{figure}



%More precisely, the domain is discretised into partitions $t_0<t_1<\dots<t_n$ and $S_0<S_1<\dots<S_n$ where $n,m\in\N$. Thus, given that the domains are equidistantly spaced, the method will also give an equidistant grid, as the one seen in \autoref{Figure:Finite_Difference}. Given that one has to iteratively determine the points of the solution one might have to condition the boundaries of the grid. For Black-Scholes, these \emph{boundary conditions}, as they are called, are given as $(S-K)^+$ for time $T$ and $0$ for $S=0$


%\subsection{Stability}
%A problem with some of the finite difference methods with regards to solving PDEs is that the methods might explode. 


%\subsection{Calibration}
%As the finite difference method can be used to describe PDEs of a model, it can also be used to calibrate models. In practice, this does however, require that the price of an option is known as every point which could be evaluated. In the case of the Black-Scholes model, the only parameter which can be calibrated is its volatility.