\chapter{Application 2}\label{Ch.6}
This chapter will utilise methods described in the previous chapters to derive the risk-neutral density function for call options on numerous underlying stocks. The risk-neutral function will be estimated by constructing a neural network which estimates the implied volatility. The derivatives of the implied volatility, required in \eqref{Eq:PartialC}, will be estimated using finite difference.

The project's code and data is available on GitHub:  \url{https://github.com/loprtq/P8}.


\subsection*{Data}
As mentioned, we will be working with numerous underlying stocks, specifically: Tesla Inc. (TSLA), Amazon Inc. (AMZN) and Alphabet Inc. Class C (GOOG). These stocks have been chosen because of their varying liquidity and vast option data. As we wish to derive the 


For these stocks we retrieve information about the stock price, strike price and time to maturity. Additionally, we set the risk-free rate to $r=0.0368$ and calculate the implied volatility according to the method seen in \eqref{Eq:IV_formula} using the function "\lstinline{EuropeanOptionImpliedVolaitlity}". 

% Filtrering

% Implied volatility plot
As seen in \autoref{Sec.Implied_Volatility}, the implied volatility surface only depends on the strike price $K$ and time to maturity $\tau=T-t$ given a price $S_t$. If we restrict our focus to GOOG and plot its implied volatility in \autoref{Figures:Figures/Pictures/Application/scatter.png} we see that it has the previously mentioned volatility smile (see \autoref{Sec.Implied_Volatility}).
\imgfig[1]{Figures/Pictures/Application/scatter.png}{Implied volatility of GOOG for $S_t=100$. For a 3D render, see }